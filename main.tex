%!TEX TS-program = xelatex
\documentclass[a4paper]{resume}
\usepackage{afterpage}
\usepackage{hyperref}
\usepackage{color}
\usepackage{xcolor}
\hypersetup{
    pdftitle=Resume - Brian LeBlanc,
    pdfauthor=Brian LeBlanc,
    pdfsubject=Resume,
    pdfkeywords={},
    colorlinks=false,
    allbordercolors=white
}

\begin{document}
\header{Brian LeBlanc}

% Fake text to add separator
\fcolorbox{white}{gray}{\parbox{\dimexpr\textwidth-2\fboxsep-2\fboxrule}{%
.....
}}

% In the aside, each new line forces a line break
\begin{aside}
    ~
    ~
    ~
    ~
    ~
  \section{Contact}
    18 William St.
    Fredericton,
    New Brunswick
    ~
    \textbf{\href{mailto:brian.l@unb.ca}{brian.l@unb.ca}}
    ~
  \section{Git}
    \href{https://github.com/omnibrian}{github.com/omnibrian}
    \href{https://gitlab.com/omnibrian}{gitlab.com/omnibrian}
    ~
  \section{LinkedIn}
    \href{https://www.linkedin.com/in/brianleblanc95/}{linkedin.com/in/brianleblanc95}
    ~
  \section{Languages}
    Python
    Bash
    ReactJS
    Ruby
    Java
    C
    ~
  \section{Tools}
    Git
    JIRA
    Jenkins
    Sceptre
    Terraform
    Chef
    Ansible
    ~
  \section{Fun Facts}
    Hosted personal BitWarden server on EC2
    ~
    Published packages on PyPi (see Personal Projects)
    ~
    Converted a Microsoft Surface Pro 3 to run Linux
    ~
\end{aside}

\section{Education}
\begin{entrylist}
  \entry
    {2013 - 2018}
    {Bachelor of Science in Software Engineering}
    {Fredericton, New Brunswick}
    {University of New Brunswick}
\end{entrylist}

\section{Work Experience}
\begin{entrylist}
  \entry
    {01/16 - Now}
    {DevOps Specialist}
    {Blue Spurs, Fredericton, New Brunswick}
    {Used CI/CD tools such as Jenkins to build CI/CD workflows for teams.\\
    Managed AWS infrastructure with CloudFormation, Sceptre, and Terraform.\\
    Designed AWS native infrastructures leveraging AWS services.\\
    Practiced Linux administration with Chef, Bash, Ansible, and Python.\\
    Participated in biweekly 24/7 on-call rotations through VictorOps.\\}
  \entry
    {01/15 - 04/15}
    {QA Analyst}
    {Q1 Labs, IBM, Fredericton, New Brunswick}
    {Learned and applied testing techniques (integration \& regression testing).\\}
  \entry
    {09/10 - 09/17}
    {Lighting Technician \& Operator}
    {Signature Sound, Fredericton, New Brunswick}
    {Practiced troubleshooting of technical issues with unfamiliar systems.\\
    Worked under pressure fixing time sensitive issues.}
\end{entrylist}

\section{Project Work}
\begin{entrylist}
  \wideentry
    {aws-lp}
    {\href{https://github.com/omnibrian/aws-lp}{github.com/omnibrian/aws-lp}}
    {CLI tool written in python to make it easier to work with AWS when using LastPass SSO.\\
    Calls LastPass to get SAML response and then calls STS AssumeRoleWithSAML.\\
    Published and available for install with pip: \href{https://pypi.org/project/aws-lp/}{pypi.org/project/aws-lp}\\}
  \wideentry
    {i3-config-builder}
    {\href{https://github.com/omnibrian/i3-config-builder}{github.com/omnibrian/i3-config-builder}}
    {Built python CLI tool without a framework using the argparse library.\\
    Published and available for install with pip: \href{https://pypi.org/project/i3-config-builder}{pypi.org/project/i3-config-builder}\\}
  \wideentry
    {AWS SNS Slack Notifier}
    {\href{https://gitlab.com/omnibrian/aws-sns-slack-notifier}{gitlab.com/omnibrian/aws-sns-slack-notifier}}
    {Built JavaScript script to forward SNS event information to Slack using AWS serverless.\\}
  \wideentry
    {AdventuresNB}
    {Frontend only: \href{https://github.com/omnibrian/adventuresnb}{github.com/omnibrian/adventuresnb}}
    {Project built for the Senior Design class (SWE4040) at UNB.\\
    Built a static hosted ReactJS frontend communicating with a Flask serverless backend.}
\end{entrylist}

\section{Volunteer Experience}
\begin{entrylist}
  \entry
    {2014 - 2016}
    {UNB RedShirts}
    {University of New Brunswick, Fredericton, New Brunswick}
    {\textbf{Orientation Leader}\\
    Gained leadership experience.}
\end{entrylist}

\section{Extracurriculars}
\begin{entrylist}
  \entry
    {2017 Fall}
    {UCOSP - Review Board}
    {\href{https://github.com/reviewboard/reviewboard}{github.com/reviewboard/reviewboard}}
    {Contributed to the open source community.\\
    Successfully implemented features for rbtools in python.}
\end{entrylist}

\end{document}
